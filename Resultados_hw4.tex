\documentclass[12pt]{article}
\usepackage{graphicx}
\usepackage[utf8]{inputenc}
%\usepackage[document]{ragged2e}

\author{David Reyes}
\title{Graficas Tarea 4 de Metodos Computacionales}
se observan con claridad las oscilaciones de una cuerda.
\begin{document}
\maketitle
\section{Extremos Fijos}
\subsection{}
\begin{figure}[htb]
\centering
\includegraphics[width=0.8\linewidth]{ExtremosFijos.jpg}
\caption{Simulacion extremos fijos}
\label{fig_ref_cycle}
\end{figure}


\section{Extremo Libres}
\subsection{}
Se observa como el sistema entra en resonancia, lo que cause que la amplitud este en aumento.
\begin{figure}[htb]
\centering
\includegraphics[width=0.8\linewidth]{ExtremosLibres.jpg}
\caption{Simulacion un extremo libre}
\label{fig_ref_cycle}
\end{figure}



\section{Oscilacion Tambor}
\subsection{}
\begin{figure}[htb]
\centering
\includegraphics[width=0.8\linewidth]{TamborPunto_medio.jpg}
\caption{Simulacion Tambor}
\label{fig_ref_cycle}
\end{figure}



\section{Matriz completa del Tambor}
\subsection{}
\begin{figure}[htb]
\centering
\includegraphics[width=0.8\linewidth]{TamborMatriz_completa.jpg}
\caption{Simulacion Tambor}
\label{fig_ref_cycle}
\end{figure}

\end{document}
